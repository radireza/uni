\documentclass{beamer}
\usepackage[latin1]{inputenc}
\usetheme{Boadilla}
\title[Make a LaTeX presentation using Beamer]{Introduction  to Beamer\\How to make a presentation with LaTeX?}
\author{Nic Hollingum}
\institute{USYD}
\date{7 Apr, 2011}
\begin{document}

\begin{frame}
\titlepage
\end{frame}


\begin{frame}{Introduction}
This is a short introduction to Beamer class.
\end{frame}

\begin{frame}{progressive 1}
\begin{itemize}
    \item The IP is ubiquitous
    \item Services with high QoS requirements   gain momentum
    \item Value lies in services
    \pause
    \item[$\Rightarrow$] Currently deployed networks need to adapt to these tendencies.
 \item[$\Rightarrow$] The IP multimedia Subsystem is seen as a promising solution for fulfilling these needs.
\end{itemize}
\end{frame}

\begin{frame}{progressive 2}
\begin{itemize}
	\item<+-> The IP is ubiquitous
	\item<+-> Services with high QoS requirements   gain momentum
	\item<+-> Value lies in services
	\item<+->[$\Rightarrow$] Currently deployed networks need to adapt to these tendencies.
	\item<+->[$\Rightarrow$] The IP multimedia Subsystem is seen as a promising solution for fulfilling these needs.
\end{itemize}
\end{frame}

\begin{frame}{progressive 3}
	\begin{overprint}
	\onslide<2->{
		\begin{itemize}
			\item Horizontal service integration
			\item Use of service enablers for several services
		\end{itemize}
	}
	\onslide<1-2>{
		\begin{center}
		\begin{enumerate}
			\item partridge in pear tree
			\item turtle doves
			\item french hens
		\end{enumerate}
		%\begin{figure}
		%\scalebox{0.50}{
		%	\includegraphics{./figures/horizontal_vs_vertical_services_integration}
		%}
		%\caption{Horizontal vs vertical service integration}
		%\end{figure}
		\end{center}
	}
	\onslide<3->{
		\begin{itemize}
		\item[$\Rightarrow$] A thigh interaction of services is possible
		\item[$\Rightarrow$] Easier and faster service development
		\end{itemize}
	}
	\end{overprint}
\end{frame} 

%\begin{frame}{progressive 2}
%\end{frame}

%\begin{frame}{progressive 2}
%\end{frame}

%\begin{frame}{progressive 2}
%\end{frame}


\end{document}
