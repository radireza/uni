\documentstyle[11pt]{article}

\author{Nic Hollingum - 308193415}
\title{Computational Geometry - Assignment 2}

\addtolength{\oddsidemargin}{-.875in}
\addtolength{\evensidemargin}{-.875in}
\addtolength{\textwidth}{1.75in}
\addtolength{\topmargin}{-.875in}
\addtolength{\textheight}{1.75in}

\begin{document}
\maketitle

\section {Multi-Range-Query}
From lectures we learnt that range-queries on 2d points and rectangles could be done in $O(\log n + k)$ time with $O(n \log n)$ pre-processing and $O(n)$ space.
In this modified version we must determine all points lying outside all n rectangles.
As it turns out this imposes little extra overhead, given the pre-sorting time.

Sort the points ($O(n \log n$) and add them to a range-tree ($O(n \log n$), note that this must only happen once for all range queries.
Next, carry out the query for one of the n rectangles, reporting all points inside the range ($O(\log n + k)$).
Simply repeat this for each rectangle.
For now we will assume this takes $O(n \log n + nk)$ time.
We maintain a list of points ``so far outside all rectangles'' which begins full, and we remove the k reported points at each iteration.
Using a balanced binary tree of points this requires at most n deletions each in $\log n$ time, hence $O(n \log n)$

This procedure reports all points lying inside a range for all ranges, and removes those points from its result.
It therefore correctly reports only those points not inside any range.

From the above description we have presumed the operation is asymptotically dominated by the $O(n \log n + nk)$ operation.
However note that the rectangles are non-overlapping.
This means that over all iterations, each point may only lie inside one rectangle and hence only be reported once.
Then the worst case for this operation is actually $O(n \log n + n)$ or $O(n \log n)$.
Reporting all poinnts outside the rectangles can take at most $O(n)$ time if all points are reported, and so is dominated by other components.
Therefore the entire operation is asymptotically bounded by its $O(n \log n)$ components (sorting, all range queries, removing reported points from result).

\section {Competing Trees}

\section {2 Point Visibility}

\section {Star-Shaped Polygons}

\section {Inverse Range-Queries}

\end{document}
