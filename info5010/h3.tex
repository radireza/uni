\documentstyle[11pt]{article}

\author{	Nic Hollingum\\
			308193415}
\title{Real World Algorithms - Homework 3}

\addtolength{\oddsidemargin}{-.875in}
\addtolength{\evensidemargin}{-.875in}
\addtolength{\textwidth}{1.75in}
\addtolength{\topmargin}{-1.875in}
\addtolength{\textheight}{2.75in}

\begin{document}
\maketitle

\section {Problem 1}
We are attempting to maximise the number of edges, whilst preventing 2 edges from joining the same node.
Expressing this as an integer programme is trivial.
Consider the matrix $A$ which is simply an edge-incidence matrix for the graph $G$, each row represents a vertex, and each column an edge.
Then for $A \Rightarrow \{ 1, 0\}^{v, e}$, $A_{v, e}$ is 1 if edge e has an endpoint at vertex v, and 0 otherwise.
Given $c$ a vector of 1's this has exactly the same format as a normal integer programme.
If some of our $x$'s was chosen such 2 edges met at a node, this would cause $A \times x$ to have at least 1 value in b greater than 1, so long as all $x$'s are positive.
Therefore clearly b must be a vector of 1's too.

When G is bipartite we note that edges from one vertex 'group' cannot meet vertices in the other.
A in this case can be rearranged (by re-labelling vertices) into upper and lower groups where each group only has a single 1 per collumn.
A square submatrix os such a matrix must either have a at most 1 1 per collumn or be situated on the barrier between partitiions and have at most a 1 above the barrier and a 1 below.
Such determinants can never exceed 1*.
Hence $A$ is totally unimodular if $G$ is bipartite.

Note that both $c$ and $b$ are vectors of 1's, the dual problem requires only the transposition of A.
In this dual, each vertex is chosen by assigning the $y$ vector to ${1, 0}$, to minimise the number of vertices chosen such that all rows are greater or equal to the vector of 1's (formerly the cost vector)
It can be described as ``find the smallest number of vertices needed such that all edges are incident to at least one vertex'', i.e. minimal vertex cover.

This dual problem tells us something interesting about maximal matching.
The number of edges selected by a mximal matching must be at most equal to the number of vertices selected by a minimal vertex cover.
This is most notably true of bipartite graphs, where the minimal cover is to simply choose the smaller of the two sides.

As for whether this is true for graphs generally, we consider the properties of the primal and dual formulations.
The programme is not guaranteed to be integral for arbitrary graphs, and nor its dual, however we do know that integral solutions must be less optimal than their linear relaxations.
Hence the property that the number of vertices in the minimal covering is equal or greater than the number of edges in the maximal matching still holds.

\section {Problem 2}
**

* I wrote a program that exhaustively checked all square submatrices for all 20x20 matrices with this property and none were found.  I am unsure why.

** I also have no idea how to do this.  It was quite easy for me to come up with $\tilde{A}$ and $b$ such that they are integral and non unimodular, and where $x$ need not take fractional values.
\end{document}
