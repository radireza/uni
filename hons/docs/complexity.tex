\documentclass{article}

\usepackage{graphicx}
\usepackage{amssymb,amsmath}
\usepackage{theorem}

\author{Nic Hollingum - 308193415}
\title{Complexity}

\begin{document}

\section{Robust Actor Assignment Problem}
The {\em Robust Actor Assignment Problem} (RAAP) can be defined as follows:
Given a set of $n$ Actors $A = \{a_1, a_2, ... a_n\}$, grouped into $g$ disjoint subsets $G_i$ where $\bigcup_{i=1}^{g}{G_i} = A$, and a set of $p$ processors $P = \{p_1, p_2 ... p_p\}$,
where each actor $a_i$ has a cost of communicating with every other $a_j$ $communicate(a_i , a_j) \geq 0$, and each processor $p_i$ has a cost $invoke(p_i, a_j) \geq 0$ of invoking actor $a_j$ on it, and each pair of processors $p_i, p_j$ has a cost $bandwidth(p_i, p_j) \geq 0$ of allowing the actors assigned to them to communicate;
find the mapping of actors to processors such that: no two actors $a_x, a_y \in G_i$ share the same processor, and the total cost of the assignment is minimised, where that cost is the sum of all $invoke(p_i, a_j)$ for processor $p_i$ invoking actor $a_j$ and the sum of all $communicate(a_x, a_y) \times bandwidth(p_i, p_j)$ for all actors $a_x$ assigned to $p_i$ and actors $a_y$ assigned to $p_j$, $i<j$.

This problem aims to capture the requirements of robust assignment on multicomputers.
The ``robustness'' previously used informally is fomalised here as the requirement for certain actors to be placed on separate processors.
Specifically duplicate actors, which must ensure correct computation even if one of the processors fails arbitrarily.
The formal ``non-overlapping-duplication'' is realised as robustness by supplying fault-monitoring overheads, however the implementation details shall not be discussed here.

Given that this section aims to prove NP-Completeness we shall now deal with a decision version of the problem.
Simply, does there exist some assignment that does not exceed our budget $b \geq 0$.

\begin{lemma}
\label{RAAPNP}
RAAP is in NP.
\end{lemma}

\begin{proof}
First we show that the cost of a guessed assignment can be determined in polynomial time, and hence whether this exceeds the budget.
Our assignment is in the form of a list of actor-processor pairings $(a_i, p_j)$, for each of the $|A| = n$ pairings sum the $invoke(p_j, a_i)$, then for each pair of pairings $(a_x, p_i), (a_y, p_j)$ sum the communication-bandwidth costs $communicate(a_x, a_y)$ multiplied by $bandwidth(p_i, p_j)$ if $i<j$.
Clearly this operation is quadratic in the number of actor-processor pairings, which is linear in the number of actors $n$.
We must also ensure the robustness constraint has not been violated.
Simply iterate over all $G_i$ and for each group, check the mapping list has not made an overlapping assignment.
We can simply return ``no'' instantly if actors have not been assigned correctly.
Again this operation is quadratic in the number of actors.
Now we simply point out that solving this problem nondeterministically means guessing possible assignments, then checking that that assignment is valid and doesnt exceed the budget in polynomial time.
\qed
\end{proof}

\section{Multiterminal Cut Problem}
Given a graph $G=(V,E)$, a set $S=\{s_1, s_2, ... s_k\}$ of $k$ specified vertices or {\em terminals}, and a positive weight $w(e)$ for each edge $e \in E$, find a minimum weight set of edges $E' \subseteq E$ such that the removal of $E'$ from $E$ disconnects each terminal from all others.

The MTC problem is discussed at length by Dahlhaus et al in \cite{dah94} and is shown to be a member of the NP-Complete problems when $k \geq 3$.
When $k = 2$ this is simply the ST-Min-Cut problem, which is solvable in polynomial time \cite{for62}.
The decision version of this problem specifies a maximul cut weight threshold $t$ and asks if there exists a cutset with total weight less than $t$.
This problem is very similar to our assignment problem informally.
We view each of the terminal nodes as processors and each nonterminal as actors, then we say that an actor has been assigned to a processor if there is a path from it to that terminal node.
We note that after the cut there is no actor which has been assigned to multiple processors, since that would require a path between 2 terminals.

\section{Complexity of RAAP}

\begin{theorem}
\label{RAAPNPC}
RAAP is NP-Complete.
\end{theorem}

\begin{proof}
From \ref{RAAPNP} we saw that RAAP was in NP.

We restrict RAAP to MTC by allowing only instances where all bandwidths are 1 and all groupings have cardinality 1 (that is, we ignore the non-overlapping constraint).
Each terminal is a processor and each nonterminal is an actor.
The cost of communication between 2 actors (nonterminals) is $w(e)$ if there is an edge $e$ between the nonterminals, and 0 otherwise.
The cost of executing an actor $v_i$ (nonterminal) on a processor $s_p$(terminal) is $\sum_{j=1}^{k}{w(s_j,v_i)} - w(s_p, v_i)$, or the sum of all weights from this node to terminals except the current terminal.
Nonterminals with no edges to terminals will cost 0 to execute, with 1 such edge of weight $w$ they will cost $w$ to execute on every processor except the one represented by that terminal across the edge.

We use such an instance to solve MTC as simply: after assignment, cut any edge between vertices assigned to different processors.
Clearly the cost of the assignment is the same as the weight of the cutset.
Nonterminals which are on seperate processors contribute $1 \times w(e)$ to the total weight, and nonterminals assigned directly contribute the weight of all edges between them and the terminal (0 if no such edge exists) except that of the processor to which they are assigned.

Filling the communication function is clearly linear in the size of $E$.
Filling the invocation time may require iterating over $E$ for every nonterminal in $V$, which is cubic in $V$ in the worst case.
Hence the transformation is polynomial.
\qed
\end{proof}

\bibliographystyle{plain}
\bibliography{biblio}

\end{document}
