\chapter{Mathematical Model}

\section{Checkpoint Recomputation}
The checkpoint recomputation (CR) method keeps a checkpoint for the last successful execution's state.
When a steady-state schedule is completed correctly the checkpoint is updated.
If it fails the system can return to the last successful state and continue.

We understand the expected steady-state execution time of a CR fault tolerant system as follows:
\begin{itemize}
	\item The makespan of a single successful execution is known statically, we shall call this $m$.
	\item For the duration of any makespan, the probability of any single processor or communication failure is known statically.
			Hence the probability that at least one component fails can be worked out, we shall call this $p$.
	\item The time required to reset the system to the last checkpointed state is known statically, we shall call this $r$.
\end{itemize}

Since we are only concerned if a single component fails, we can take the generous presumption of failure independence (only the first failure matters).
This makes calculating the total expected execution time $t$ simple.
\begin{align}
	\nonumber t_i & = (1-p)m + p(m + r + t_{i+1}) \\
	\nonumber t & = t_0 \\
	\nonumber \mbox{Which can be re-written as:}& \\
	\nonumber t & = \left({(m+r)\displaystyle\sum\limits_{i=0}^\infty p^i }\right) - r \\
	\nonumber & = (m+r)\left({{1}\over{1-p}}\right) - r\\
	\nonumber & = {{m+rp}\over{1-p}}
\end{align}