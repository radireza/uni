\chapter{NP-Hardness}
\label{chapHardness}

In this chapter we formalise the problem of assigning actors to processors.
This problem is called the {\em Robust Actor Allocation Problem} ({\em RAAP}).
We reduce it to the well-known {\em Multi-Terminal Cut} problem ({\em MTC}) to demonstrate its NP-Hardness.
This leads us into considering a heuristic which will be discussed in Chapter \ref{chapAlgos}.

\section{Graph and Path Predicate}

We must define these constructs for use later.
These are the standard definitions, and are put here for convenience.

\begin{definition}
Graph $G$
\begin{align}
	\nonumber G = & (V,E) \\
	\nonumber V = & \{v_1, v_2, ..., v_n\} \quad & (vertices)\\
	\nonumber E \subseteq & V^2 & (edges)
\end{align}

We shall deal exclusively with so-called ``undirected'' graphs, in which for all $(u,v) \in E : (u,v) = (v,u)$.

Note: if $V' \subseteq V$ or $E' \subseteq E$ then we may call the graph $G'=(V',E')$ a \emph{subgraph} of $G=(V,E)$.
\end{definition}

\begin{definition}
\label{defHardPath}
Path Predicate.

The predicate $\path{E'}{u}{v}$ is true for all $E' \subseteq E$ and $u, v \in V$ in the graph $G=(V,E)$ in the following cases:

\begin{align}
	\nonumber \forall (u,v) \in V^2 : \path{E}{u}{v} \Leftrightarrow & ( u = v \\
	\nonumber & \vee \langle \exists w \in V : \path{E}{u}{w} \wedge (w, v) \in E \rangle).
\end{align}

Note also then that:

\begin{align}
	\nonumber \forall u, v \in V: & \path{E}{u}{v} \Leftrightarrow \path{E}{v}{u} \\
	\nonumber \forall u, v, w \in V: & \path{E}{u}{w} \wedge \path{E}{w}{v} \Rightarrow \path{E}{u}{v}.
\end{align}

For simplicity we shall use $\npath{E}{u}{v}$ when the predicate does not hold.

\end{definition}

\section{RAAP and MTC}

Next we formalise the problems that we are dealing with.

\begin{definition}
{\em Multi-Terminal Cut}

Given a graph $G=(V,E,w,T)$, a set $T=\{t_1, t_2, ..., t_k\}$ of $k$ specified vertices or {\em terminals} : $T \subseteq V$, and a positive weight $w(e)$ for each edge, find a minimum weight set of edges $E' \subseteq E$ such that the removal of $E'$ from $E$ disconnects each terminal from all others.
An instance of MTC is therefore defined formally:

\begin{align}
	\nonumber MTC = & (V,E,w,T)\\
	\nonumber & E \subseteq V^2\\
	\nonumber & w : E \rightarrow \mathbb{Z}^+\\
	\nonumber & T \subseteq V
\end{align}

Solution: $\mSolution \in 2^E$

Solution Cost: $\mCost : 2^E \rightarrow \mathbb{Z}^+$
\begin{align}
	\nonumber \mCost(S) = \displaystyle\sum\limits_{(u,v) \in S} w(u,v)
\end{align}

Feasible Solution Space: $F \subseteq 2^E$
\begin{align}
	\nonumber \forall S \in F : \forall s, t \in T: s \neq t \Rightarrow \npath{E \setminus S}{s}{t}
\end{align}

Optimal Solution Space: $O \subseteq F$
\begin{align}
	\forall S_o \in O : \forall S_f \in F : \nonumber C(S_o) \leq C(S_f)
\end{align}

\end{definition}

\begin{definition}
The {\em residual graph} ($G_r$) is a subgraph of MTC graph $G=(V,E,w,T)$ with some subset of edges $S \subseteq E$ removed:

$G_r = (V, E \setminus S, w, T)$
\end{definition}

\begin{definition}
{\em Robust Actor Allocation}

Given a graph $G=(V,E,w_c,w_i, P, A)$, a partitioning of $V$ into $k$ subsets and a set of processors $P$, where $w_c : E \rightarrow \mathbb{Z}^+$ and $w_i : V \times P \rightarrow \mathbb{Z}^+$, find a mapping $\rSolution : V \rightarrow P$ such that
the cost function $\rCost(\rSolution)$ (detailed below) is minimised and no actors in the same partition are assigned to the same processor.

\begin{align}
	\nonumber RAAP = & (V,E,w_c, w_i, P, A)\\
	\nonumber & E \subseteq V^2\\
	\nonumber & w_c : E \rightarrow \mathbb{Z}^+\\
	\nonumber & w_i : V \times P \rightarrow \mathbb{Z}^+\\
	\nonumber & P = \{p_1, p_2, ...p_p\} \\
	\nonumber & A \subseteq 2^V \quad s.t. : \displaystyle\bigcup\limits_{S \in A} S = V \wedge \langle \forall S_1, S_2 \in A : S_1 \neq S_2 \Rightarrow S_1 \cap S_2 = \emptyset \rangle
\end{align}

Solution: $\rSolution \in V \rightarrow P$

Solution Cost: $\rCost : V \rightarrow P \rightarrow \mathbb{Z}^+$
\begin{align}
	\nonumber \rCost(\rSolution) = \displaystyle\sum\limits_{(u,v) \in E : \rSolution(u) \neq \rSolution(v)} w_c(u,v) + \displaystyle\sum\limits_{v \in V} w_i(v, \rSolution(v))
\end{align}

Feasible Solution Space: $\rFeasible \subseteq V \rightarrow P$
\begin{align}
	\nonumber \forall \rSolution \in \rFeasible & \\
	\nonumber \mbox{(1)} \quad & \rCost(\rSolution) \neq \infty \quad \mbox{We use $\infty$ as a placeholder for a very large integer} \\
	\nonumber \mbox{(2)} \quad & \forall A_i \in A : \forall u , v \in A_i : u \neq v \Rightarrow \rSolution(u) \neq  \rSolution(v)
\end{align}

\noindent The feasibility constraint (1) here is used as a means of excluding parts of the solution space known to be non-optimal from the feasible region too.
It is simply a shortcut but does not affect the results.

Optimal Solution Space: $\rOptimal \subseteq \rFeasible$
\begin{align}
	\nonumber 	\forall \rSolution_o \in \rOptimal : \forall \rSolution_f \in \rFeasible : \rCost(\rSolution_o) \leq \rCost(\rSolution_f).
\end{align}

\end{definition}

\section{Mapping}

To begin our reduction, we show the relationship between the two problems.
This section defines the mapping, which is used to convert arbitrary instances of the MTC problem, into RAAP instances.
The properties of this mapping will be examined in later sections.

\begin{definition}
\label{defHardMap}
Mapping ($\mapsto$)

\begin{align}
	\nonumber I \in & MTC(V,E,w,T) &\\
	\nonumber I' \in & RAAP(V', E', w_c, w_i, P, A) &\\
	\nonumber & I \mapsto I' if and only if: &\\
	\nonumber V' & = V\\
	\nonumber E' & = E\\
	\nonumber P & = T\\
	\nonumber w_c & = w\\
	\nonumber w_i(v,p) & = \mbox{$\left\{ 
		\begin{array}{l l}
			0 \quad & \mbox{if $v \not\in T$}\\
			0 & \mbox{if $v \in T$ and $p = v$}\\
			\infty & \mbox{otherwise}\\ \end{array} \right.$} \\
	\nonumber A & = \{\{v\} | v \in V\}
\end{align}

Note that $w_i(v, p)$ may take infinity where $v$ is a terminal that doesnt map to $p$.
This is simply a way of forcing all terminals to be mapped to themselves, as we shall see.

We have used ``$=$'' to keep future descriptions simple, where we really mean that there is a function $f : V \rightarrow V'$ such that
\begin{align}
	\nonumber & V' = \{f(v) | v \in V\} , \\
	\nonumber & E' = \{(f(u), f(v)) | (u,v) \in E\} , \\
	\nonumber & P = \{f(t) | t \in T\} , \mbox{and} \\
	\nonumber & \langle \forall (u, v) \in E : w_c(f(u), f(v)) = w(u, v) \rangle
\end{align}
\end{definition}

Note that the mapping \ref{defHardMap} constrains $A$ to be singletons of vertices, which makes it impossible to violate feasibility constraint (2).
When talking about mapped instances we may therefore ignore this constraint.

\section{Feasibility of the Mapping}
\label{secHardFeas}

Referring to Definition \ref{defHardMap} we must see if the mapping allows us to determine feasible solutions to the original MTC problem ($\mInstance$) when we find a feasible solution to the mapped RAAP instance ($\rInstance$).
To do this we must have some mechanism to retrieve MTC solutions from the mapped RAAP solution.

\begin{definition}
Reversal Function $\reverse$.
Given an assignment function $\rSolution$, $\reverse$ returns a list of edges $(u, v)$ when $u$ and $v$ are differently assigned.

\begin{align}
	\nonumber \reverse & : V \rightarrow P \rightarrow 2^E \\
	\nonumber \reverse(\rSolution) & = \{(u,v) \in E | \rSolution(u) \neq \rSolution(v)\}
\end{align}
\end{definition}

\begin{lemma}
\label{EDGEASSIGN}
The two nodes at the endpoint of any edge in the residual graph were assigned to the same processor:

\begin{align}
	\nonumber \forall(u, v) \in E \setminus \reverse(\rSolution) : \rSolution(u) = \rSolution(v).
\end{align}
\end{lemma}

\begin{proof}
Immediate by definition of $\reverse$.
\end{proof}

\begin{lemma}
\label{PATHASSIGN}
The two nodes for which the path predicate is true in the residual graph were assigned to the same processor:

\begin{align}
	\nonumber \forall (u, v) \in V^2 : \path{E \setminus \reverse(\rSolution)}{u}{v} \Rightarrow \rSolution(u) = \rSolution(v).
\end{align}
\end{lemma}

\begin{proof}
We assume that $\path{E \setminus \reverse(\rSolution)}{u}{v}$ holds, now we establish that in such cases $\rSolution(u) = \rSolution(v)$ by induction.

By definition \ref{defHardPath}, when $\path{E \setminus \reverse(\rSolution)}{u}{v}$ holds there can be two cases:

\begin{align}
	\nonumber \mbox{1)} \quad & u = v \quad \mbox{(base case)} \\
	\nonumber & \quad \mbox{obviously $\rSolution(u) = \rSolution(v)$} \\
	\nonumber \mbox{2)} \quad & \exists w \in V: \path{E \setminus \reverse(\rSolution)}{u}{w} \wedge (w, v) \in E \setminus \reverse(\rSolution)
\end{align}

\noindent We may deduce that $\rSolution(u) = \rSolution(w)$ inductively.
Also, since $(w, v) \in E \setminus \reverse(\rSolution)$, from Lemma \ref{EDGEASSIGN} we have that $\rSolution(w) = \rSolution(v)$, and therefore that $\rSolution(u) = \rSolution(v)$.
\end{proof}

\begin{lemma}
\label{FORCEASSIGN}
Any feasible assignment $\rSolution$ maps all terminals to themselves:

\begin{align}
	\nonumber \forall \rSolution \in F' : \forall t \in T : \rSolution(t) = t.
\end{align}
\end{lemma}

\begin{proof}
Immediate by the definition of the mapping.
Otherwise $\rCost(\rSolution) = \infty$ and therefore $\rSolution \not\in \rFeasible$.
\end{proof}

\begin{lemma}
\label{REVERSEFEASABLE}
The reverse solution for any feasible $\rSolution$ is also a feasible solution for the MTC instance.

\begin{align}
	\nonumber \forall \rSolution \in \rFeasible : \reverse(\rSolution) \in \mFeasible
\end{align}
\end{lemma}
\begin{proof}
In other words, we must show that $\forall s, t \in T : s \neq t \Rightarrow \npath{E \setminus \reverse(\rSolution)}{s}{t}$.

To prove this by contradiction let us presume $\exists \beta \in \rFeasible : \reverse(\beta) \not\in \mFeasible$.
Therefore $\exists s, t \in T : s \neq t \wedge \path{E \setminus \reverse(\beta)}{s}{t}$.

Now, let us assume $s, t \in T, s \neq t$ and $\path{E \setminus \reverse(\beta)}{s}{t}$.
First from \ref{FORCEASSIGN}, it is the case that $\beta(s) = s$ and $\beta(t) = t$.
So since $s \neq t$ then $\beta(s) \neq \beta(t)$.

Second from \ref{PATHASSIGN}:
\begin{align}
	\nonumber \forall (u, v) \in V^2 : \path{E \setminus \reverse(\rSolution)}{u}{v} & \Rightarrow \rSolution(u) = \rSolution(v)
\end{align}

And therefore since $\path{E \setminus \reverse(\beta)}{s}{t}$, $\beta(s) = \beta(t)$.

But this is a contradiction.
\end{proof}

\section{Optimality of the mapping}
\label{secHardOpt}

We need to show that the mapping allows us to solve the MTC problem optimally if we can solve the RAAP mapped instance optimally.
To do this we must show that all optimal solutions of the RAAP instance $\rInstance$ can be reversed using $\reverse$ to optimal solutions of the MTC instance $\mInstance$.

We intuit a surjection on the reversal function.
If we were to name processors in the assignment, we can have a different assignment by switching the names of two processors that would yield the same reversed solution.
This intuition brings us to look at a way of retreiving at least one $\rSolution$ from a given $\mSolution$ and working with that.

We do this using an {\em inversion function} $\inverse$, which has some useful properties.
Most importantly we can say something about the cost of inverted solutions which we use to prove the optimality proposition.

\begin{definition}
Inversion Function $\inverse$

First we pick arbitrarily $t_0 \in T$ as a default processor.
We use it to be able to assign vertices $v \in V'$ for which $\forall t \in T : \npath{}{v}{t}$.
\begin{align}
	\nonumber \inverse & : 2^E \rightarrow (V \rightarrow P) \\
	\nonumber \inverse(S) & = \rSolution \mbox{ where } \forall v \in V' \rSolution(v) = \mbox{$\left\{ 
		\begin{array}{l l}
			t \quad & \mbox{if $t \in T \wedge \path{(V,E \setminus S)}{v}{t}$}\\
			t_0 & \mbox{otherwise}\\ \end{array} \right.$} 
\end{align}

\end{definition}

\begin{lemma}
\label{INVERSESUBSET}
Any edge in the inversion of a feasible solution to a multiterminal cut with endpoints assigned to different processors had to be in the solution.

\begin{align}
	\nonumber & \forall \mSolution \in \mFeasible : \forall (u, v) \in E' : \inverse(S)(u) \neq \inverse(S)(v) \Rightarrow (u,v) \in S \\
	\nonumber \mbox{or} &\\
	\nonumber & \forall S \in F : \{(u,v) |(u, v) \in E' \wedge \inverse(S)(u) \neq \inverse(S)(v)\} \subseteq S
\end{align}
\end{lemma}
\begin{proof}
To prove this by contradiction we must show that there is no edge with differently assigned endpoints that is not in the solution.
Assume $\exists (u,v) \in E' : \inverse(S)(u) \neq \inverse(S)(v) \wedge (u,v) \not\in S$.

For the above assumption we are concerned with two possible cases, either neither $u$ nor $v$ has been assigned the default processor (1) $t_0$ or one of them has (2).
It is possible that one of the nodes happened to be assigned the default processor because they are connected (rather than because it was disconnected from every terminal) but we can ignore this case as we only need to show the Lemma for all differently assigned nodes.

1) When neither node was assigned the default processor we have the following:
\begin{align}
	\nonumber \exists s \neq t \in T  & : \path{E \setminus S}{u}{s} \wedge \path{E \setminus S}{v}{t} \\
	\nonumber \exists (u, v) \in E \setminus S& \\
	\nonumber & \Rightarrow \path{E \setminus S}{u}{t} \wedge \path{E \setminus S}{v}{s} \\
	\nonumber & \Rightarrow S \not\in F
\end{align}

2) When one of the nodes was assigned the default processor we have the following:
\begin{align}
	\nonumber \mbox{Let: } \inverse(S)(u) = t_0 & \\
	\nonumber \exists s \in T \setminus \{t_0\} & : \path{E \setminus S}{v}{s} \\
	\nonumber & \Rightarrow \path{E \setminus S}{u}{s} \\
	\nonumber & \Rightarrow \inverse(S)(u) = s = t_0
\end{align}
But that is a contradiction, since we assumed $v$ was assigned to a different terminal to $t_0$ and therefore $s \neq t_0$.

Hence both cases would lead to contradictions.
\end{proof}

\begin{lemma}
\label{INVERSEFEASIBLE}
Any feasible solution to a multiterminal cut is also a feasible solution to its inverted mapped instance:

\begin{align}
	\nonumber \forall S \in F : \inverse(S) \in F'
\end{align}
\end{lemma}
\begin{proof}
Since the mapping allows us to ignore the partitioning constraint on feasibility we are only concerned with the cost.

Immediately from the path predicate and definition of $\inverse$ we see that $\forall t \in T : \inverse(S)(t) = t$.

\begin{align}
	\nonumber \rCost(\inverse(\mSolution)) & = \displaystyle\sum\limits_{(u,v) \in E' : \inverse(S)(u) \neq \inverse(S)(v)} w_c(u,v) + 0 \\
	\nonumber \rCost(\inverse(\mSolution)) & \leq \displaystyle\sum\limits_{(u,v) \in E'} w_c(u,v) \\
	\nonumber \rCost(\inverse(\mSolution)) & \neq \infty \\
	\nonumber \inverse(\mSolution) & \in F'
\end{align}
\end{proof}

\begin{lemma}
\label{INVERSECOST}
The inverted solution to a multiterminal cut problem costs the same or less than the cut.

\begin{align}
	\nonumber \forall S \in F : C(S) \geq C'(\inverse(S))
\end{align}
\end{lemma}
\begin{proof}
\begin{align}
	\nonumber C(S) & = \displaystyle\sum\limits_{(u,v) \in S} w(u,v)\\
	\nonumber C'(\inverse(S)) & = \displaystyle\sum\limits_{(u,v) \in E' : \inverse(S)(u) \neq \inverse(S)(v)} w_c(u,v) + \displaystyle\sum\limits_{v \in V'} w_i(v, \inverse(S)(v)) \\
	\nonumber & = \displaystyle\sum\limits_{(u,v) \in E' : \inverse(S)(u) \neq \inverse(S)(v)} w_c(u,v) + 0 \\
	\nonumber \mbox{And from \ref{INVERSESUBSET}} \\
	\nonumber & \{(u,v) | (u,v) \in E' \wedge \inverse(S)(u) \neq \inverse(S)(v)\} \subseteq S \\
	\nonumber & \Rightarrow \displaystyle\sum\limits_{(u,v) \in S} w(u,v) \geq \displaystyle\sum\limits_{(u,v) \in E' : \inverse(S)(u) \neq \inverse(S)(v)} w_c(u,v) \\
	\nonumber & \Rightarrow C(S) \geq C'(\inverse(S))
\end{align}
\end{proof}

\begin{lemma}
\label{REVERSECOST}
The solution to the RAAP instance of a mapped MTC problem costs the same as the reverse solution.

\begin{align}
	\nonumber \forall \rSolution \in F' : C'(\rSolution) = C(\reverse(\rSolution))
\end{align}
\end{lemma}
\begin{proof}
Immediately by the definitions of the mapping, $\reverse$ and Lemma \ref{FORCEASSIGN}.

From \ref{FORCEASSIGN} we know $\forall t \in T : \rSolution(t) = t$ hence $\displaystyle\sum\limits_{v \in V'} w_i(v, \rSolution(v)) = 0$.

Then:
\begin{align}
	\nonumber \rCost(\rSolution) & = \displaystyle\sum\limits_{(u,v) \in E' : \rSolution(u) \neq \rSolution(v)} w_c(u,v) + 0 \\
	\nonumber \mCost(\reverse(\rSolution)) & = \displaystyle\sum\limits_{(u,v) \in E' : \rSolution(u) \neq \rSolution(v)} w(u,v) \\
	\nonumber \rCost(\rSolution) & = \mCost(\reverse(\rSolution))
\end{align}
\end{proof}

\begin{lemma}
\label{REVERSEOPTIMAL}
Any optimal solution to RAAP can be reversed to give an optimal solution to MTC.

\begin{align}
	\nonumber \forall \rSolution \in O' : \reverse(\rSolution) \in O
\end{align}
\end{lemma}
\begin{proof}
This uses a contradiction that would arise otherwise because of lemmas \ref{INVERSEFEASIBLE}, \ref{REVERSECOST} and \ref{INVERSECOST}.
\begin{align}
	\nonumber \mbox{Assume:} \quad & \exists \rSolution \in O' : \reverse(\rSolution) \not\in O \\
	\nonumber & \Rightarrow \exists S \in F : C(S) < C(\reverse(\rSolution)) \quad \mbox{Definition of optimal space} \\
	\nonumber \\
	\nonumber \rSolution \in \rOptimal & \Rightarrow \forall \beta \in \rFeasible : \rCost(\rSolution) \leq \rCost(\beta) \quad (*)\\
	\nonumber \\
	\nonumber \mbox{from \ref{INVERSEFEASIBLE}}\quad & \inverse(\mSolution) \in \rFeasible \\
	\nonumber \mbox{from \ref{INVERSECOST}}\quad & \rCost(\inverse(\mSolution)) \leq \mCost(S) \\
	\nonumber & \mCost(S) < \mCost(\reverse(\rSolution)) \\
	\nonumber \mbox{from \ref{REVERSECOST}}\quad & \mCost(\reverse(\rSolution)) = \rCost(\rSolution)\\
	\nonumber & \Rightarrow C'(\inverse(S)) < C'(\rSolution) \\
	\nonumber & \Rightarrow \rSolution \not\in O'
\end{align}
But this contradicts (*).

\end{proof}

\section{Complexity}

Sections \ref{secHardFeas} and \ref{secHardOpt} show that there exists a relationship between the problems.
We have a polynomial time mechanism to convert instances of the MTC problem to the RAAP.
We also have a polynomial time algorithm to retrieve the solution to the original MTC problem from a solution to the RAAP.
The retrieved solution is feasible if it was a feasible solution to the RAAP, and optimal in the same fashion.

This proves the reduction, and shows that the RAAP is at least as hard as MTC.
Hence:
\begin{theorem}
The RAAP is in the class of NP-Hard problems.
\end{theorem}
\begin{proof}
By reduction from MTC.
\end{proof}