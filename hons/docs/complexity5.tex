\documentclass{article}

\usepackage{graphicx}
\usepackage{amssymb,amsmath}
\usepackage{theorem}

\author{Nic Hollingum - 308193415}
\title{Complexity}

\begin{document}

\begin{definition}
Graph $G$
\begin{align}
	\nonumber G = & (V,E) \\
	\nonumber V = & \{v_1, v_2, ..., v_n\} \quad & (vertices)\\
	\nonumber E \subseteq & V^2 & (edges)
\end{align}
\end{definition}

\begin{definition}
Path $\pi$ for graph $G = (V,E)$.

\begin{align}
	\nonumber \forall v \in V & : \pi(v, v) = \epsilon \\
	\nonumber \forall (u,v) \in E & : \pi(u, v) = \{(u, v)\} \\
	\nonumber \forall u, v \in V & : \pi(u, v) = \pi(u, w) + \pi(w, v) \quad \mbox{Here we use ``$+$'' to denote concatenating subpaths}
\end{align}

If a path $\pi(u, v)$ cannot be formed by successively adding edges then the path is said to not exist, so there is no path etween disconnected nodes $u$ and $v$.
\end{definition}

\begin{definition}
Pathset $\Pi_G$ for graph $G = (V,E)$.

\begin{align}
	\nonumber \Pi_G(v_a, v_b) = \{\pi(v_a, v_b) : v_a, v_b \in V\} \quad or \quad \emptyset \quad \mbox{if there is no path from $v_a$ to $v_b$}
\end{align}

Note: $\forall v \in V : \Pi(v, v) = \{\epsilon\} \neq \emptyset$.
\end{definition}


\begin{definition}
{\em Multi-Terminal Cut}

Given a graph $G=(V,E,w,T)$, a set $T=\{t_1, t_2, ... t_k\}$ of $k$ specified vertices or {\em terminals} : $T \subseteq V$ , and a positive weight $w(e)$ for each edge, find a minimum weight set of edges $E' \subseteq E$ such that the removal of $E'$ from $E$ disconnects each terminal from all others.
An instance of MTC is therefore defined fromally:

\begin{align}
	\nonumber MTC = & (V,E,w,T)\\
	\nonumber & E \subseteq V^2\\
	\nonumber & w : E \rightarrow \mathbb{Z}^+\\
	\nonumber & T \subseteq V
\end{align}

Solution Space: $2^E$

Solution Cost: $C : 2^E \rightarrow \mathbb{Z}^+$
\begin{align}
	\nonumber C(S) = \displaystyle\sum\limits_{(u,v) \in S} w(u,v)
\end{align}

Feasible Solution Space: $F \subseteq 2^E$
\begin{align}
	\nonumber \Pi_{(V,E \setminus S)}(t_i, t_j) = \emptyset \quad \forall (t_i, t_j) \in T^2 \setminus \{(t, t) | t \in T\}
\end{align}

Optimal Solution Space: $O \subseteq F$
\begin{align}
	\forall S_j \in F , \quad \forall S_i \in O \quad \nonumber C(S_i) \leq C(S_j)
\end{align}

\end{definition}

\begin{definition}
Given a solution $S \in 2^E$ to MTC(V,E,w,T) the {\em Residual Graph} ($G_r$) is defined as:

$G_r = (V, E \setminus S)$
\end{definition}

\begin{definition}
{\em Robust Actor Allocation}

Given a graph $G=(V,E,w_c,w_i)$ and a set of processors $P$, where $w_c : E \rightarrow \mathbb{Z}^+$ and $w_i : V \times P \rightarrow \mathbb{Z}^+$, find a mapping $\alpha : V \rightarrow P$ such that:
$\displaystyle\sum\limits_{(u,v) \in E : \alpha(u) \neq \alpha(v)} w_c(u,v) + \displaystyle\sum\limits_{v \in V} w_i(v, \alpha(v))$ is minimised.

\begin{align}
	\nonumber RAAP = & (V,E,w_c, w_i, P, A)\\
	\nonumber & E \subseteq V^2\\
	\nonumber & w_c : E \rightarrow \mathbb{Z}^+\\
	\nonumber & w_i : V \times P \rightarrow \mathbb{Z}^+\\
	\nonumber & P = \{p_1, p_2, ...p_p\} \\
	\nonumber & A = \{A_i | a_i \subseteq V\}
\end{align}

Solution Space: $\alpha \in P^V \subseteq P(V \times P)$

Solution Cost: $C : P^V \rightarrow \mathbb{Z}^+$
\begin{align}
	\nonumber C(\alpha) = \displaystyle\sum\limits_{(u,v) \in E : \alpha(u) \neq \alpha(v)} w_c(u,v) + \displaystyle\sum\limits_{v \in V} w_i(v, \alpha(v))
\end{align}

Feasible Solution Space: $F \subseteq P^V$
\begin{align}
	\nonumber C(\alpha) & \neq \infty \quad \mbox{We use $\infty$ as a placeholder for a very large integer} \\
	\nonumber \forall A_i \in A : \forall u \neq v \in A_i : \alpha(u) & \neq  \alpha(v)
\end{align}

Optimal Solution Space: $O \subseteq F$
\begin{align}
	\nonumber 	\forall \alpha_i \in O : \forall \alpha_j \in F : C(\alpha_i) \leq C(\alpha_j)
\end{align}

\end{definition}

\begin{definition}
Mapping

For convenience we shall call the feasible and optimal solution spaces and cost for the RAAP side of the mapping $F'$, $O'$ and $C'$ respectively.

\begin{align}
	\nonumber I \in & MTC(V,E,w,T) &\\
	\nonumber I' \in & RAAP(V', E', w_c, w_i, P, A) &\\
	\nonumber & I \mapsto I' if: &\\
	\nonumber V' & = V\\
	\nonumber E' & = E\\
	\nonumber P & = T\\
	\nonumber w_c & = w\\
	\nonumber w_i(v,p) & = \mbox{$\left\{ 
		\begin{array}{l l}
			0 \quad & \mbox{if $v \not\in T$}\\
			0 & \mbox{if $v \in T$ and $p = v$}\\
			\infty & \mbox{otherwise}\\ \end{array} \right.$} \\
	\nonumber A & = \emptyset
\end{align}

In the case of $V', E', P$ and $w_c$ we have used ``$=$'' to denote a bijection.
More formally we mean: $Y = X \Rightarrow \exists f : X \rightarrow Y \quad s.t. \quad \forall x_i \neq x_j \in X : f(x_i) \neq f(x_j) \wedge f(x_i), f(x_j) \in Y$

The above mapping makes $A = \emptyset$ which makes it impossible to violate feasibility this way.
When talking about mapped instances we may therefore ignore it.
\end{definition}

\begin{lemma}
\label{FORCEASSIGN}
Given a feasible solution $\alpha \in F'$ to the mapped RAAP instance $I'$ for an MTC problem $I$:

\begin{align}
	\nonumber \forall v \in P : \alpha(v) = v
\end{align}
\end{lemma}

\begin{proof}
\begin{align}
	\nonumber \alpha \in & F' \\
	\nonumber C'(\alpha) \neq & \infty \\
	\nonumber \displaystyle\sum\limits_{(u,v) \in E' : \alpha(u) \neq \alpha(v)} w_c(u,v) + \displaystyle\sum\limits_{v \in V'} w_i(v, \alpha(v)) \neq & \infty \\
	\nonumber \displaystyle\sum\limits_{v \in P} w_i(v, \alpha(v)) \neq & \infty \\
	\nonumber \forall v \in P : \alpha(v) \neq & \infty \\
	\nonumber \forall v \in P : \alpha(v) = & v
\end{align}
\end{proof}

\begin{definition}
Reversal Function $\gamma$

\begin{align}
	\nonumber \gamma & : (P^V) \rightarrow P(E) \\
	\nonumber \alpha & \mapsto \{(u,v) \in E | \alpha(u) \neq \alpha(v)\}
\end{align}
\end{definition}

\begin{lemma}
\label{PATHASSIGN}
Given a feasible solution $\alpha \in F'$ to the mapped RAAP instance $I'$ for an MTC problem $I$ and the residual graph $G_r$ from $I$ made using $S = \gamma(\alpha)$ :
\begin{align}
	\nonumber \forall v_a, v_b \in V : if & \quad \Pi_{G_r}(v_a, v_b) \neq \emptyset \\
	\nonumber & \Rightarrow \alpha(v_a) = \alpha(v_b)
\end{align}
\end{lemma}

\begin{proof}
This derives from the definition of $\gamma$.
We proove it inductively on the path.
Let $G_r = (V, X)$ and $\pi \in \Pi_{G_r}(v_a, v_b)$ :

\begin{align}
	\nonumber & X = E \setminus \gamma(\alpha) : \pi \subseteq X \\
	\nonumber & \forall (x_i, x_{i+1}) \in \pi : (x_i, x_{i+1}) \not\in \gamma(\alpha) \\
	\nonumber & \Rightarrow \alpha(v_i) = \alpha(x_{i+1}) \\
	\nonumber & \forall (x_i, x_{i+1}), (x_{i+1}, x_{i+2}) \in \pi : (x_i, x_{i+1}), (x_{i+1}, x_{i+2}) \not\in \gamma(\alpha) \\
	\nonumber & \Rightarrow \alpha(x_i) = \alpha(x_{i+1}) = \alpha(x_{i+2}) \\
	\nonumber \mbox{hence, let} & \quad \{(v_a, x_i), (x_i, x_{i+1}),...,(x_{j-1}, x_j), (x_j, v_b)\} = \pi \\
	\nonumber & \Rightarrow \alpha(v_a) = \alpha(x_i) = \alpha(x_{i+1}) = \alpha(x_{j-1}) = \alpha(x_j) = \alpha(v_b)
\end{align}

Here we use $\subseteq$ to denote that all components of a path are edges in the residual graph

\end{proof}

\begin{lemma}
Given a feasible solution space $F'$ to the mapped RAAP instance $I'$ for an MTC problem $I$ with feasible solution space $F$.
\begin{align}
	\nonumber \forall \alpha \in F' : \quad \gamma(\alpha) \in F
\end{align}
\end{lemma}

\begin{proof}
We show by contradiction that no solution $\beta$ for $I'$ which maps to infeasible $2^E \setminus F$ can exist.

Let $G_r$ be the residual graph from $I$ made using $S = \gamma(\beta)$.

\begin{align}
	\nonumber \gamma(\beta) & \in 2^E \setminus F \\
	\nonumber & \exists \quad t_1\neq t_2 \in T : \Pi_{G_r}(t_1, t_2) \neq \emptyset \\
	\nonumber & \exists \quad \pi \in \Pi_{G_r}(t_1, t_2) \\
	\nonumber & \beta(t_1) = \beta(t_2) & \mbox{(from \ref{PATHASSIGN})}\\
	\nonumber \\
	\nonumber & \beta(t_1) = t_1, \beta(t_2) = t_2 & \mbox{(from \ref{FORCEASSIGN})} \\
	\nonumber & t_1 \neq t_2 \\
	\nonumber & \beta(t_1) \neq \beta(t_2)
\end{align}

\end{proof}

\bibliographystyle{plain}
\bibliography{biblio}

\end{document}