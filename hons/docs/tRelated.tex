\chapter{Related Work}
\label{chapRelated}

\section{SDF and Parallel Paradigms}

\subsection{Historic Work}

The Synchronous Dataflow (SDF) paradigm builds on the more general computation graphs described in \cite{kar66}.
At the time that model was only a graph-theoretical representation of parallel computation, and was more of an abstraction rather than a paradigm.
We compare this with the more fleshed-out description of the fundamentals of the language in \cite{sdfBook}.
Synchronous Dataflow is an amalgamation of previous input on the computational graphs issue, and is considered the defacto description of the paradigm.
We are particularly interested in the calculation of actor repetitions described first in Lee and Messerschmitt's \cite{lee87} and appearing in \cite{sdfBook}.
This is important because it is the static calculation of demand and supply by actors (called blocks in \cite{lee87}) that allows SDF to be optimized statically, giving it advantages over dynamic systems such as S-Net in \cite{pen09}.

Dynamic dataflow systems like S-Net, are designed for solving certain classes of problems pure SDF cannot.
This is seen to be one of the major problems with StreaMIT, as stated by Thies, Karczmarek and Amarasinghe \cite{thies02}.
SDF requires static declaration of actor consumptions and productions on each of its channels, which allows the calculation of Lee's fixed periodic schedule.
Thies and Amarasinghe argued that systems which allow the same level of abstractions without static IO rates are difficult to optimize for multiprocessor environments \cite{thies10}.
Implementations of S-Net described in \cite{pen09} are not sufficiently fault tolerant for our purposes.
Without knowing these rates it is unable to assign more duplications of certain actors which are known to be in high demand.

S-Net and StreaMIT are not the only dataflow implementations, as a much earlier work on LUSTRE can be found in \cite{cas87}.
LUSTRE is declarative as opposed to functional (StreaMIT).
LUSTRE implements several novel notions of computation (sequence operators, clocking) that normal declarative languages do not deal with, which are required in implementing a streaing paradigm.

The notion of data-dependent clocks is an important one.
In LUSTRE data precedence is defined by an incremental counter (called the clock) assigned in a tuple with each data item.
We contrast this to the distributed clocking system presented by Lamport in \cite{lam78}.
Both Lamport and Caspi et al. strike the distinction between physical clocks and logical clocks, so as to be able to deal with the notion of precedence in distributed systems.
This is a necessary distinction, since it does not actually matter what the physical time of an execution is, but only the time relative to the execution of other distributed components.
Lamport defines a {\em happened before} relationship among processes, which imposes a partial ordering of the system at the level of processes.

Given a partial ordering of data items and processes on distributed systems in general and SDF-like systems specifically, we can exploit the parallelism inherent in SDF.
A means of doing this is to assign actors to processors, in a similar fashion to \cite{par03}.
The distributed process networks described in that paper scale well with instance size, given sufficiently good load balancing.
However, that work only deals with worker count, assuming data independence and stateless processes.
The paper therefore deals more with buffer management and load balancing, in order to achieve {\em embarrasing parallelism}.
Embarassingly parallel programs are programs with no data interdependencies (i.e. unordered) and no process states (in our case, statless actors).
This assumption is not allowed in general for SDF programs, whose actors can have arbitary state and whose tokens are necessarily ordered.
This is not a problem for the most well known massively-parallel paradigm, MapReduce.
MapReduce \cite{dea08} is a popular tool for large scale data-crunching.
Much of the work goes into distributed-memory grid/cloud systems, however implementations such as PHOENIX \cite{ran07} demonstrate the paradigm's viability for shared-memory multithreaded systems.

\subsection{Modern work}

More modern implementations of streaming systems build on the simpler ideas expressed above.
These systems are designed in an era dominated by real world peta-scale systems involving thousands of computers, and so can be better tested under those conditions.
StreaMIT \cite{thies01, thies02, thies10} we saw earlier, but other systems such as pig \cite{ols08}, SPADE \cite{ged08}, Sawzall \cite{pik05} and DryadLINQ \cite{yu08} also exist in this domain.

The trend is in data analysis and distributed database applications.
Olston et al. describe Pig Latin \cite{ols08} which is a declaritive-style language for large-scale distributed database and compiles into execution plans for the open source Map-Reduce implementation {\em hadoop}.
This is very similar in principle to Sawzall \cite{pik05} by Pike et al. and SPADE \cite{ged08} by Gedik et al., though Sawzall is designed more for automatic analysis whereas SPADE is for more general databasing.
It is also worthwhile to note that pig and Sawzall are system agnostic whereas SPADE is designed for IBM's System-S.
For comparison, DryadLINQ \cite{yu08} by Yu et al. incorporates features of general purpose languages (objects, strong typing) as well as highly parallel execution environments (Map-Reduce, distributed databases).
To an extent these systems attempt to provide farmiliar programming concepts to users along with well-known powerful technologies.
In contrast the more rigid and limiting StreaMIT is not so extensible, instead focusing more on performance through compiler optimisations \cite{thies01}.

The notion of optimising streaming and peta-scale computations is not new, however, an important means of formalising these notions for compiler validation and optimisation came from Soul{\'e} et al. with their universal calculus \cite{sou10}.
This allows us to compare the genrality and performance of streaming languages themselves, and tthe work (besides introducing the calculus {\em Brooklet}) also compares Sawzall, StreaMIT and CQL, a dialect of StreamSQL.
Understandably then, a large part of these modern streaming systems' focus is on performance optimizations, specifically linear or near-linear scaling improvements (both for the system itself and for the computations).
To a extent the literature on these systems neglects issues such as faultiness in favour of performance.

\section{Fault Tolerance}

Fault tolerance is the main focus of our work.
The notion of fault tolerance is not new \cite{ran75}, however, it has become absolutely essential in more recent cloud and grid systems.

The motivation behind this problem was described by Reed et al in \cite{ree06}.
They point out that even ``under the generous assumption of fault independence'' that systems with $10^4$ nodes each having a lifetime of $10^6$ hours will likely fail in as little as 100 hours.
The work demonstrates both software and hardware changes to provide fault-tolerance, specifically dealing with an extension to the MPI framework to allow automatic checkpointing and recovery.
Though the chance of an individual failure is high, given enough backup processors the time to failure can be made sufficiently long \cite{ree06}.
Unfortunately this implementation makes extensive use of hardware fault-detection systems, which we presume are unavailable in our context.
This paper can be seen as presenting the checkpointing method of ensuring fault-tolerance, that is, if a computation fails the computation is backed-up to a checkpoint and re-run, which is one of the fault-tolerance mechanisms we shall be examining.

Randell's early work \cite{ran75} focused more on software redundancies in the case of software faults.
The work did discuss the notion of formal redundancies and checkpointing for fault-tolerance, which are techniques we are employing.

Lamport et al. \cite{lam86} describe the notion of a Byzantine fault, in order to provide tolerance to ``arbitrary'' faults.
The definition of a fault is discussed in more depth, we presume that each of the processors is malicious (that is, a traitorous general of the Byzantine army), and may act so as to prevent the army from taking action.
This description accurately captures the nature of faults, that they may cause the computation to halt, to produce incorrect results, or to crash completely.
The paper deals with assuring fault tolerance in the context of distributed clock synchronisation, however, the principles are applicable to our problem of protecting against (specifically) node failure.
The important observation is that there may be no more than $n/3 - 1$ faulty for n total.
Lamport proposes a voting scheme \cite{lam86} to determine correct computations from incorrect ones.

In this thesis we also examined static fault tolerance via task replication.
An implementation of such a system for mobile grid computing is presented by Litke et al. in \cite{lit07}.
The work discusses a technique to determine the number of replicas required in order to assure fault tolerance of a system given a certain probability of failure.
Similarly to us, Litke et al. provide simulated results based on a Knapsack formulation of the scheduling of such fault-tolerant systems.
This work is very similar to ours, except it does not deal with the problems unique to SDF, namely actor indivisibility, nor the communication costs of the system.

Few of the implementations we have seen so far deal explicitly with fault-tolerance.
S-Net makes no mention of fault tolerance or recovery mechanisms in \cite{pen09}.
The MapReduce implementations \cite{dea08, ran07} both state their fault-tolerance mechanisms, namely reassigning jobs if a worker is presumed to have failed.
Google's implementation discusses the master controller pinging each worker and presuming the worker fails if the ping times out.
However neither system deals with the posibility (more likely in the case of PHOENIX \cite{ran07}) of master-failure.
Google's implementation simply aborts the MapReduce computation if the master fails.
Granted master failure is unlikely, however a better fault-tolerance scheme would be more desirable for this feature of robustness.
Further, none of these systems describe the relationship between faults and makespan in the way of Litke et al. \cite{lit07}.